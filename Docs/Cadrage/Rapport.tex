\documentclass[a4paper, 12pt, titlepage, oneside, french]{article}

\usepackage[francais]{babel}
\usepackage[utf8]{inputenc}
\usepackage{fancyhdr}
\usepackage{graphicx}
\usepackage{amsmath}
\usepackage{amssymb}
\usepackage[backend=biber]{biblatex}
\usepackage{float}
\usepackage{subcaption}

\graphicspath{ {images/}}
\addbibresource{Biblio.bib}

\author{Martin Olivier, Fabien Goglio}
\title{Rapport InSite}

\pagestyle{fancy}

%Put the chapter and section header in lowercase
%\renewcommand{\chaptername}[1]{%
%	\markboth{#1}{}}

\renewcommand{\sectionmark}[1]{%
	\markright{\thesection\ #1}}

	%delete the current header
\fancyhf{}

\fancyhead[LE, RO]{\bfseries\thepage}
\fancyhead[LO]{\bfseries\textit\rightmark}
%\fancyhead[RE]{\bfseries\textit}
\fancyhead[RE]{WAT}
\renewcommand{\footrulewidth}{0.5pt} %space for the rule
\fancypagestyle{plain}{\fancyhead{}, %get rid of header on plain pages
	\renewcommand{\headrulewidth}{0pt} % and the line
}


\begin{document}
\begin{titlepage}
	\centering
		\vfill
		    {\bfseries\Large
			Rapport de Mi-Parcours\\
			InSite\\
			Janvier 2018\\
			\vskip2cm
			Martin OLIVIER, Fabien GOGLIO\\
		    }    
		\vfill
		\includegraphics[width=8cm]{Logo_Preview.png}
		\begin{figure}[b]
			\includegraphics[width=4cm]{Logo-ESIEA.jpg}

		\end{figure}
		\vfill
		\hfill {\bfseries\Large
		 Suiveur:\\
		 \hfill Sylvie ZAGO}
\end{titlepage}

\pagenumbering{roman}
\newpage
	\tableofcontents
\newpage
\cleardoublepage
\pagenumbering{arabic}
\section{Remerciement}
	Nous tenons a remercier \[...\]
	\newpage
\section{Introduction}
	\newpage
\section{Présentation}
	\subsection{Description}
		InSite analyse et détecte des objets archéologiques dans des relevés géomagnétiques, en utilisant des techniques novatrices dans le domaine de
		la Computer Vision, ainsi que de l'apprentissage automatique.
	\subsection{Objectifs}
	Nous cherchons a:
	\begin{itemize}
		\item Analyser automatiquement des relevés géomagnétiques 
		\item Produire des cartes d'intérêts, en indiquant les zones ou des objets archéologiques pourrait se trouver
		\item Détecter et réduire le bruit présent dans l'image
	\end{itemize}

	\newpage

\section{Cadrage}
	\subsection{Budget}
	Étant d'abord et surtout un projet de recherche d'informatique, InSite ne requiert pas de budget. Nous avons pu obtenir les relevés magnéto-métrique sans cout, et les ressources nécessaires pour mener a bien ce projet sont déjà a notre disposition. 
	\subsection{Dates clefs}
	\subsection{Organisation}
	InSite se compose de 2 étudiants, Fabien GOGLIO et Martin OLIVIER. Par cette taille réduite, l'organisation est simplifie. Les rôles sont distribue selon l'intérêt de pour la tache, et selon sa compétence a mener cette tache a bien. Les recherches sur les techniques a utiliser, et le développement d'outil et de code sont faites de manières indépendantes, même si une communication constante est faite. 
	\subsubsection{Démarche}
	Comme inSite est un projet non pas de développement, mais de recherche, nous utilisons une démarche différentes de celle trouve habituellement dans les PST: nous passons une grande partie de notre temps a faire des recherches, et nous n'écrivons que des petits scripts, qui répondent a une tache particulière.   
	\subsection{Planning}
	\subsection{Langages et technologies utilise}
	Pour simplifier le développement, nous utilisons Python2 avec les librairies numpy, pyplot, cv2, qui possèdent de nombreuse fonctions mathématique et d'analyse d'image déjà implémentées.
	\newpage

\section{Démarche Initiale}
	\subsection{Carte de relevés géomagnétiques}
	\subsection{Détections d'objets}
	Nous avons décidé de débuter nos recherches de détections de formes par des solutions simples, avec l'attente qu'elle nous donnent des résultats de bonne qualité. L'attrait de ces méthodes étaient principalement qu'elles ne requiert pas de quantité de données importante, et qu'elles étaient a notre porte d'un point de vue académique. Nous avons donc approche le problème de deux façons différentes:
	\begin{itemize}
		\item Filtrage
		\item Algorithmes de détections de bords.
	\end{itemize}
	\subsubsection{Filtrage}
	Nous avons cherché à "nettoyer" les images : atténuer le bruit, très présent dans certaines zones de l'image, et assez handicapant pour la détection de bord.
	\begin{figure}[H]
		\centering
		\includegraphics[width=5cm]{filter_input.png}
		\caption{Exemple d'image bruitée}. 
		\label{fig:FilterInput}
	\end{figure}
	\pagebreak
	Nous avons pour cela commencé par appliquer les techniques classiques de réduction de bruit : 
	\begin{itemize}
		\item Flou gaussien
		\item Filtres coupe-bande plus complexes (approche fréquentielle)
		\item Dilatation/Érosion
	\end{itemize}
	\paragraph{\textbf{Flou gaussien}}
	La manière la plus classique d'atténuer le bruit d'une image est le flou gaussien. Cette technique consiste à remplacer chaque pixel par la moyenne des ses voisins, avec une distance paramètrable.
	\begin{figure}[H]
		\centering
		\includegraphics[width=6cm]{filter_gaussian.png}
		\caption{Image traitée avec un flou gaussien de 15 pixels}. 
		\label{fig:FilterGaussian}
	\end{figure}

	\pagebreak
	\paragraph{\textbf{Approche fréquentielle}}
	Ici, l'idée est d'appliquer le filtre sur directement sur la transformée de fourier, affin de pouvoir choisir précisément les fréquences que l'on souhaite conserver ou supprimer.
	L'approche la plus concluante à été d'appliquer un passe-bande pour supprimer les hautes fréquences (qui contiennent le bruit), ainsi que les basses fréquences pour augmenter le contraste.
	\begin{figure}[H]
		\centering
		\includegraphics[width=\linewidth]{filter_fft.png}
		\caption{Différentes étapes du traitement}. 
		\label{fig:FilterFFT}
	\end{figure}
	On voit sur la figure \ref{fig:FilterFFT} le traittement séquentiel de l'image (de gauche à droite, puis de haut en bas) : on calcule la FFT, on applique un masque (le coupe-bande), puis on applique la FFT inverse. \\
	Cette approche peut donner des résultats intéressants, mais le problème est que la bande de fréquence à garder dépend des images. On perd donc de vue l'objectif initial, : l'automatisation.

	\pagebreak
	\paragraph{\textbf{Dilatation/Érodation}}
	La dilatation et l'érosion sont des traitements qui travaillent sur la morphologie de l'image.
	La dilatation "dilate" les formes : elle augmente la superficie des structures.
	À l'inverse, l'érosion réduit la superficie des structures, elle "érode" les bords.
	\begin{figure}[H]
		\centering
		\begin{subfigure}[b]{0.3\linewidth}
			\includegraphics[width=\linewidth]{filter_dilate-erode_ex-base.png}
			\caption{Image originelle}
		\end{subfigure}
		\begin{subfigure}[b]{0.3\linewidth}
			\includegraphics[width=\linewidth]{filter_dilate-erode_ex-dilate.png}
			\caption{Dilatation}
		\end{subfigure}
		\begin{subfigure}[b]{0.3\linewidth}
			\includegraphics[width=\linewidth]{filter_dilate-erode_ex-erode.png}
			\caption{Érosion}
		\end{subfigure}
		\caption{Exemple de dilatation et d'érosion sur une image simple}. 
		\label{fig:FilterDilateErodeEx}
	\end{figure}
	L'érosion permet donc de supprimer les petites formes, nous pouvons donc l'utiliser pour effacer notre bruit.
	Pour compenser l'effet de l'érosion sur le reste de l'image, nous appliquons aussi une dilatation : le bruit est complètement effacé, et le reste est restauré par la dilatation.

	\begin{figure}[H]
		\centering
		\includegraphics[width=6cm]{filter_dilate-erode.png}
		\caption{Image traitée par 2 érosions suivies de 2 dilatations}. 
		\label{fig:FilterDilateErode}
	\end{figure}

	\pagebreak
	\paragraph{\textbf{Conclusion}}
	Les techniques classiques de réduction de bruit dans une images semblent donner des résultats prometteurs : visuellement, l'image est plus lisible. Le résultat n'est cependant pas satisfaisant car, d'une part, le bruit reste présent, mais d'autre part -- et c'est le plus important -- le bruit n'est pas sufisemment effacé pour une détection de contours, comme nous allons le voir juste après.

	\pagebreak

	\subsubsection{Algorithmes de détections de bords}
	Nous avons débuté par l'utilisation de l'opérateur de Sobel. Cet technique produit des images en noir et blanc ou les bords ont une valeur de blanc élevé. Un bord est défini comme un endroit de l'image ou la magnitude de son gradient est élevé. Son fonctionnement est décrit en plus de détails ci-dessous: 
		\paragraph{\textbf{Recherche du gradient d'intensité de l'image}}
				On applique un kernel de Sobel: il s'agit simplement de deux matrices 3x3 permettant de calculer la dérivé première verticale et horizontale par convolution. Ces deux matrices sont:
				\\ \[G_x = \begin{bmatrix}  +1  & 0 & -1 \\ +2 & 0 & -2  \\ +1 &  0 & -1\end{bmatrix} 
					G_y = \begin{bmatrix} +1 & +2 & +1  \\  0 & 0 &  0  \\ -1 & -2 & -1\end{bmatrix}\]
						A chaque point de l'image on détermine la magnitude de ce gradient par la formule
					\[\nabla(G) = \sqrt{G_x^2 + G_y^2}\]
	\paragraph{\textbf{Création d'une image des bords}}
	Une fois cet opération réalisée, il nous suffit d'associer la valeur du gradient a une valeur de gris. Une grande magnitude du gradient, indiquant que le pixel orignal appartenait a un bord donnera un pixel blanc, tandis qu'une faible intensité donnera un pixel noir.
	Sur des images "propres" on obtient ce genre de résultats: \\
	\begin{figure}[]
		\centering
		\begin{subfigure}[b]{0.4\linewidth}
			\includegraphics[width=\linewidth]{ValveOriginal.png}
			\caption{Image originelle}
		\end{subfigure}
		\begin{subfigure}[b]{0.4\linewidth}
			\includegraphics[width=\linewidth]{ValveSobel.png}
			\caption{Image traite avec Sobel}
		\end{subfigure}
		\caption{Comparaison d'une image sans traitement avec celle traite avec Sobel, de Wikipedia \cite{WikiCannyOriginal}\cite{WikiSobel}}. 
		\label{fig:SobelGood}
	\end{figure}
	On observe que les détails de l'image disparaissent, et que les bords, c'est a dire les endroits ou la magnitude du gradient de l'image sont important sont visible par des nuances de gris.\\
	Lorsque l'on applique l'opérateur Sobel sur nos images, nous obtenons, dans les meilleurs des cas, ces résultats:
	\begin{figure}[]
		\centering
		\begin{subfigure}[b]{0.4\linewidth}
			\includegraphics[width=\linewidth]{Sobel1b.png}
			\caption{Image originelle}
		\end{subfigure}
		\begin{subfigure}[b]{0.4\linewidth}
			\includegraphics[width=\linewidth]{Sobel1a.png}
			\caption{Image traite avec Sobel}
		\end{subfigure}
		\label{fig:OurSobel}
	\end{figure}

	Sur nos images, les résultats obtenus ne sont pas très pertinent. Le bruit est trop prévalent, et les bords des objets que nous cherchons a détecter sont trop faibles pour que l'opérateur de Sobel puisse donne des résultats intéressant.

	Malgré des résultats peu encouragent avec Sobel, nous avons décidé d'utiliser un autre algorithme, celui de Canny. L'algorithme de Canny reprend les matrices de Sobel, et applique quelques opérations supplémentaire pour obtenir des bords plus mieux défini. Les détails de l'algorithme sont décrit ci-dessous:
	\begin{enumerate}
		\item \textbf{Réduction du bruit:}\\
			\indent On applique un filtre Gaussien 5x5 pour réduire le bruit présent dans l'image
		\item \textbf{Recherche du gradient d'intensité de l'image:}\\  
			\indent On applique ensuite un kernel de Sobel sur l'image "lisse" dans les directions verticales et horizontales afin d'obtenir les dérivés premières dans
			la direction verticale $G_x$ et horizontales $G_y$. Ce procédé est identique a celui décrit plus haut.

		\item \textbf{Suppression des non maximums locaux}\\
			\indent Une fois les gradients obtenus, on analyse tout les pixels de l'image, et on détermine si le pixel est un maximum local dans la
			direction du gradient. \\
			Si oui, c'est un bord et sa valeur est garde pour la prochaine étape, sinon, elle est mise a 0. On obtient une image binaire, avec que des bords

		\item \textbf{Seuil d'Hysterisis} \\
			\indent On utilise deux seuils, $minVal$ et $maxVal$. Tout les bords ayant une intensité de gradient supérieur a $maxVal$ est forcement un
			bord, ceux en dessous de $minVal$ sont forcement des non-bords, et sont donc abandonne. Les bords qui sont entre ces deux seuil sont classe
			"bords" ou "non-bords" selon leur connectivité. Si ils sont connecte a des pixels qui sont des forcement des bords, alors ce sont des bords,
			sinon, ils sont aussi abandonne.\\

	\end{enumerate}
	Cet algorithme étant déjà disponible dans la librairie d'analyse de d'image PythonCV2, nous n'avons pas eu a l'implémenter, et nous avons bénéficié de quelques optimisations rendant l'exécution plus rapide.
	Avec une image "propre", on obtient ce genre d'image:
	\newpage
	\begin{figure}[] %%NE PAS OUBLIER DE LINK L'AUTEUR : By Simpsons contributor, CC BY-SA 3.0, https://commons.wikimedia.org/w/index.php?curid=8904364
		\centering
		\begin{subfigure}[b]{0.4\linewidth}%CORIGER LA POSITION DES FIGURES
			\includegraphics[width=\linewidth]{ValveOriginal.png}
			\caption{Image originelle}
		\end{subfigure}
		\begin{subfigure}[b]{0.4\linewidth}
			\includegraphics[width=\linewidth]{ValveCanny.png}
			\caption{Image traite avec Canny}
		\end{subfigure}
		\caption{Comparaison d'une image sans traitement avec celle traite avec Canny, de Wikipedia \cite{WikiCannyOriginal}\cite{WikiCanny}}. 
		\label{fig:cannyGood}
	\end{figure}
	On observe que les bords sont bien plus défini, et propre par rapport a ceux obtenu avec l'opérateur Sobel. Mieux, les bords "majeurs", sont préservé, alors que les bords "mineurs" disparaissent. Cela est du aux deux dernières étapes de l'algorithme de Canny.
			Lorsque l'on applique Canny a nos images, nous obtenons ceci:
	\begin{figure}[!h]%CORRIGER LA POSITION DES FIGURES
		\centering
		\begin{subfigure}[b]{0.4\linewidth}
			\includegraphics[width=\linewidth]{Canny1a.png}
			\caption{Image originelle}
		\end{subfigure}
		\begin{subfigure}[b]{0.4\linewidth}
			\includegraphics[width=\linewidth]{Canny1b.png}
			\caption{Image traite avec Canny}
		\end{subfigure}
		\label{fig:OurCanny}
	\end{figure}

	Même si il y a une amélioration certaine par rapport a Sobel, l'algorithme ne parvient toujours pas a trouver les contours des objets d'intérêt, encore une fois a cause du a la grande quantité de bruit que l'on trouve dans l'image. Cependant  on observe une densité de points plus élevé aux niveau des endroits d'intérêts, comme ici, et une densité faible dans les zones ou on ne trouve ni objets ni bruit.

	\begin{figure}[]%CORRIGER LA POSTION DES FIGURES
		\centering
		\begin{subfigure}[b]{0.4\linewidth}
			\includegraphics[width=\linewidth]{CANNY_ExempleDetailsA.png}
			\caption{Image originelle}
		\end{subfigure}
		\begin{subfigure}[b]{0.4\linewidth}
			\includegraphics[width=\linewidth]{CANNY_ExempleDetailsB.png}
			\caption{Image traite avec Canny}
		\end{subfigure}
		\caption{Comparaison entre de détails entre une image sans traitement avec peu de bruit, et l'image traite avec Canny. Même si les bords ne sont pas détecté correctement, les lieux d'intérêt de l'image (la forme circulaire) ont très visiblement une haute densité de points.}
		\label{fig:CannyDetails}
	\end{figure}





	\subsubsection{Problèmes rencontres}
	De toute évidence, nous avions atteint un obstacle: le bruit. Celui ci était présent sur toutes les images, au point d'empêcher toute tentative d'analyse classique de signal. La magnétométrie est très sensible aux objets métalliques, et dans la région ou ont été fait les relevés, de nombreux objets métalliques contemporains sont présent dans le terrain, notamment a cause des combats de la 2\textsuperscript{ème} guerre mondiale, mais aussi a cause des labours, et du déchets métallique produit par l'occupation humaines de ces terrains.\\
	Une analyse classique ne suffisait donc pas, et le développement de méthodes de filtrage et de détection de bords étant capable de traiter des images aussi bruite est hors de notre porte. Notre attention devait donc se porter sur la résolution d'un problème, annexe a celui qui nous intéressait, mais dont les techniques développé par sa résolution pouvait nous aider dans notre objectifs principal.

\newpage
\section{Développement d'une nouvelle méthodologie}
	\subsection{Détections de bruits par réseaux neuronaux}
	Comme dit plus haut, les techniques classiques utilise dans l'analyse d'image sont inapplicables dans notre cas, du notamment au haut niveau de bruit présent dans les images a traites. Cependant, il est évidents que ce bruit n'est pas identique aux objets que nous cherchons, car sinon, les archéologues ne pourrait pas analyser ces images. Pour mieux diriger nos recherches nous avons choisi de détecter non pas un objets archéologique, mais un type de bruit très prévalent dans ces images: un dipôle. Les différences entre ce bruit et le reste de l'image étant trop subtils pour y établir un ensemble de règles prédéfinies, nous avons décidé d'utiliser des réseaux de convolution, qui sont un type de réseaux neuronaux, très adaptes a l'analyse d'image pour détecter automatiquement le bruit, dans l'espoir de non seulement nettoyer le bruit, avec par exemple un flou Gaussien, mais également de détecter les objets archéologiques.
	\subsubsection{Caractérisation}
	Les dipôles proviennent des objets métalliques contemporains, et se caractérisent par un pôle positif très fort au niveau de l'objet, suivi d'un halo circulaire négatif, dont le diamètre est approximativement 2 fois celui de l'objet. On peut voir sur la figure \ref{fig:DipoleExample} que les anomalies se ressemblent beaucoup, et son très présentes sur virtuellement toutes les cartes. De plus elles sont assez grandes pour être immédiatement reconnu sur nos relevés, ce qui simplifie la tache de labelisation des exemples a apprendre. 
	%RAJOUTER IMAGE DIPOLE REEL
	%RAJOUTER PLOT3D DIPOLE
	\begin{figure}[]%CORRIGER LA POSTION DES FIGURES
		\centering
		\begin{subfigure}[b]{0.3\linewidth}
			\includegraphics[width=\linewidth]{DipoleExemple1.png}
		\end{subfigure}
		\begin{subfigure}[b]{0.3\linewidth}
			\includegraphics[width=\linewidth]{DipoleExemple2.png}
		\end{subfigure}
		\begin{subfigure}[b]{0.3\linewidth}
			\includegraphics[width=\linewidth]{DipoleExemple3.png}
		\end{subfigure}

		\caption{Quelques exemples de dipôles, récupéré sur plusieurs relevés différents. On peut observer leur similitudes, tant par leur formes que par leur tailles}
		\label{fig:DipoleExample}
	\end{figure}

	\subsubsection{Explication de l'approche}
	Les réseaux de convolution sont une innovation récente dans le champs de l'apprentissage automatique. Le premier exemple d'un réseau de convolution moderne provient de Yann Le Cun \cite{lecun-01a} en 1998. En 2012, les réseaux de convolution réapparaitront avec AlexNet \cite{NIPS2012_4824}, et restent aujourd'hui sur le devant de la scène de l'apprentissage automatique. 
	Les réseaux de convolutions, que nous appellerons CNNs (Convolutionnal Neural Network) se composent de 2 parties
	%EXPLICATION CNNS
	\subsubsection{Problèmes initiaux}
	\subsubsection{Solutions}
	\textbf{Génération d'exemple}:\\
	Classifier a la main tout les exemples nécessaires a l'entrainement du réseau était une tache surhumaine. Par soucis d'efficacité, nous avons décidé de construire nous même
	nos exemples, et des les utiliser en combinaison avec des exemples provenant des images, et classifie a la main.
	Pour avoir une meilleure chance de créer un bruit proche de la réalité, nous avons utilise 2 approches différentes:
	\begin{itemize}
		\item Une approche informatique
		\item Une approche mathématique
	\end{itemize}
	L'approche informatique consiste a créer 2 cercles, de tailles variables,un noir et un blanc, ayant la même origine. Ces cercles sont imprime sur une \textit{backplate} grise avec du bruit génère aléatoirement. Afin d'améliorer le résultat, du bruit est ajoute aux cercles, augmentant particulièrement aux bords de ceux ci.

	Nous avons génère une fonction mathématique se rapprochant de celle d'un dipôle.\\
	Il s'agit d'une fonction sinusoïdale amortie exponentiellement modifie pour que sont domaine soit $\mathbb{R}^2 \to \mathbb{R}$:
	\[f(x,y) = A.e^{-\lambda . \sqrt{(x+i)^2+(y+j)^2}}.\cos(\omega . \sqrt{(x+i)^2+(y+j)^2} + \phi)\]
	Avec:\\
	\indent $A$ l'amplitude initiale, choisie aléatoirement entre 200 et 255\\
	\indent$\lambda$ coefficient d'amortissement\\
	\indent$\omega$ vitesse angulaire\\
	\indent$\phi$ angle de phase a l'origine\\
	\indent$i,j$ décalage horizontal et vertical, respectivement, par rapport a l'origine

	Avec la génération aléatoire des paramètres de cette fonction, nous pouvons obtenir une grande variété d'exemple qui ressemble aux dipôles retrouves dans nos cartes.\\ \\ 
	%TODO: Rajouter exemple 3D Dipole Reel vs Dipole Genere
	\textbf{Traitement des images}:\\
	\indent Afin d'obtenir une meilleure qualité d'exemples nous avons utilise ceux provenant de nos données originelle. Nous avons d'abord découpe chacune des images en carre de taille 96*96 pixels; en décalant le carre de 48 pixels verticalement et horizontalement a chaque fois, nous pouvons obtenir encore plus d'exemples. Au final, nous avons pu obtenir plus de 51000 carres, grâce au fait que les images était très grande (souvent plus de 2000*2000 pixels). Nous avons ensuite, a la main, trouve tous les carres contenant un dipôles. %RAJOUTER NOMBRE EXEMPLE ET PLUS DE PRECISIONS %  
	%TODO : Parler de la preparation du dataset (decoupage, ajout de bruit sur nos exemples reel etc)
	\\ \\
	\textbf{Implémentation Réseau Neuronal}\\
	\indent Nous avons implémenté un réseau neuronal classique s'entrainant sur des exemples générés. Le réseau reprends l'architecture de l'exemple de TensorFlow pour la classification de chiffres manuscrit avec une modification du nombre d'entrée pour s'adapter a la taille des exemples générés. Lors de l'entrainement, on a pu observer une très rapide maximisation du taux de détection, arrivant a un taux de réussite de 100\% sur les exemples de test en 4 époques. Nous avons ensuite tente d'appliquer ce réseau sur une carte, dans l'espoir de créer une \textit{heatmap} d'activation du réseau. Malheureusement, ce réseau possédait un taux d'erreur très élevé sur les exemples réels. Nous ne possédions pas le temps d'implémenter un réseau neuronal de convolution, et les résultats obtenu avec le réseau neuronal "classique" nous indiquait qu'utiliser seul des exemples génères ne suffirait pas pour obtenir un bon taux de détection sur les données réelles. Nous avons donc préférée utiliser notre temps restant a construire une meilleure base d'exemple en mixant données réelle et données générée, pour obtenir non seulement une meilleure variance, mais un \textit{dataset} plus proche de la réalité.
\newpage
\section{Bilan provisoire et perspectives}
	\subsection{Conclusion} %Changer le nom ?
	Ce semestre fut riche en travail et en résultat. Nous avons énormément appris sur le travail d'analyse de données
	\subsection{Implémentation d'un réseau de convolution}
	La prochaine étape est évidement l'implémentation d'un réseau de convolution, apprenant sur le dataset. La partie la plus chronophage, celle de préparer la base de données étant déjà accompli, cette tache ne prendra que peu de temps. 
	\subsection{Choix de la direction de recherche}

\newpage
\section{Annexes}

\medskip
\newpage
\printbibliography
\end{document}
